A practical solution to location privacy should be incrementally deployable. We claim it should hence reconcile the \emph{economic} value of location to aggregators, usually ignored by prior works, with a user's \emph{control} over her information.
Location information indeed is being collected and used by many mobile services to improve revenues, and this gives rise to a heated debate: Privacy advocates ask for stricter regulation on information collection, while companies argue that it would jeopardize the thriving economy of the mobile web. 

We describe a system that gives users control over their information and does not degrade the data given to aggregators. 
Recognizing that the first challenge is to express locations in a way that is meaningful for advertisers and users, we propose a \emph{keyword} based design. 
Keywords characterize locations, let the users inform the system about their sensitivity to disclosure, and build information directly usable by an advertiser's targeting campaign. 
Our work makes two main contributions: we design a market of location information based on keywords and we analyze its robustness to attacks using data from ad-networks, geo-located services, and cell networks.
 % to analyze the economic consequences of such a system on ad revenue; and we conduct a small scale experiment of the system to collect preliminary results on the behavior of real users if location information markets were deployed.


% key things:
% 1) easy to use
% 2) user control
% 3) user incentive
% 4) doesn't reduce value
% 5) preserves privacy

% 7/15/2013 version
% Location information is increasingly being collected and used by the majority of mobile services, arguably for service personalization. 
% However, the widespread usage of this data for improving targeted advertisements extends way beyond its intended use and potentially reveals user's location information to multiple third parties. 
% Not surprisingly, this has opened a heated debate: Privacy advocates ask for stricter regulation on information collection, while companies argue that it would jeopardize the thriving economy of the mobile web. 
% Indeed, mobile privacy solutions often fail to gain traction as users judge it profitable to disclose some location information in exchange for compensation, which is found in the form of a free convenient service. 
% A comprehensive privacy solution, in contrast, should ideally allow users to opt-in for some information disclosure when they find it profitable. 
% In this paper, we propose and evaluate a new location privacy system to reconcile the economic value of user information to the advertiser with the need for users' control. 
% Recognizing that the first challenge is to express location in a way that is meaningful for advertisers and users, we propose a keyword based design. 
% Keywords characterize locations, let the users inform the system about their sensitivity to disclosure, and build information directly usable by an advertiser's targeting campaign. 
% Our work makes three contributions. First, we present a design for implementing a market of location information and discuss its security implications. 
% Nest, we use data from ad-networks, geo-located services, and mobile traces from both online applications and records of cellular networks to analyze the potential economic consequences of such system on the revenue of advertising. 
% Finally, we conduct a small scale experiment of the system to collect preliminary results on the behavior of real users in real location information markets.


% OLD
% Location information is increasingly being collected and used to personalize mobile services.
% Additionally, mobile services increase their advertising revenues by improving targeting with this information.
% Not surprisingly, this exchange raises many privacy concerns, with advocates and regulators asking for stricter laws governing the collection of location information.
% Many previous works that attempt to alleviate these concerns either fail to account for the economic value of a user's personal information or create a system that is too complicated for the typical user to understand and use.
% In this paper, we present a simple new system of location monitoring done in a privacy-sensitive way which gives control to the users while retaining value for advertisers.
% The main idea is to let users with mobile devices opt-in to sell their location information for monetary compensation.
% Location information is sold in a format that is both intuitively easy to grasp and yet well understood in terms of monetization: the keyword.
% In our scheme, each location is associated with a keyword.
% Users decide on keywords for which they feel comfortable sending their informtaion to advertisers.
% Buyers of user information gain access to users only at locations associated with keywords that are not privacy sensitive.
% Beyond describing our proposed design, we analyze both the system's economic implications as well as the security considerations.
% Additionally, we build a small-scale deployment of the system, conduct a study with real users, and characterize the results.
% Finally, we analyze the privacy and economic consequences of applying the system to real cost-per-click values from ad-networks, keywords from an online review website, public Foursquare check-ins and a massive dataset of anonymized mobility patterns from cell phone call description records.


% OLDER
%because it enhances the users' experience. More generally, it allows to better profile a user based on where she has been, and 
%it also provides context on the user's immediate environment. 
%Today, even applications that do not use your location have no relation 
%to your location continue to track where you are, at least to increase revenue obtained from advertising third party. 
% In this paper, we present a solution to location privacy driven by economics. 
% The main idea is to let users with mobile devices opt-in to
% sell their location information for monetary compensation.
% Buyers then gain access to these users at locations that are not privacy sensitive.
% We call our solution transactional location privacy and it consists of two major components. 
% The first is an economic model that determines the value of location information. We develop a model
% of quantifying this value and use two large datasets -- over a million publicly available checkins from Foursquare and anonymized mobility patterns of millions 
% of users from two large European cities -- to analyze our model. We find that the value of 
% location information is highly skewed -- the top 5\% of the locations
% account for more than 60\% of the total value. 
% This suggests that most locations have low economic value, hence need not be
% disclosed. 

% The second component is a proposal for an architecture that protects a user's privacy.
% We use a blacklisting technique and a mapping of locations to keywords to achieve this.
% We use a system that determines a location's sensitivity based on keywords associated with it.
% The user creates a blacklist of keywords describing locations where they do not want to be tracked.
% When the 	and advertisers receive no indication of the user's presence.
% The second component is to preserve location privacy for the locations that the user does not want released. 
% We use a combination of location matching and a blacklist (that prevents sensitive locations to released) to achieve this. 
% Again, 
% using real anonymized data we investigate the economic value generated in such a system and the likely revenue lost 
% due to a user's privacy settings. %versus the content of the blacklist and 
% We show that there is a 
% sweet spot in terms of privacy %that can be preserved (in terms of content of the blacklist) and the
% and the monetary value that can obtained from the
% relatively low-sensitivity information released. We describe an architecture to achieve this. 
%In this paper, we study for the first time the consequences of disclosing in real time your location from a joint economic-privacy perspective . While most previous algorithms aims at guaranteeing privacy condition for highest accuracy, we aim at guaranteeing these constraints for maximum value, in our case through a simplified model of advertising. We propose an incentive scheme to achieve this value while aligning users and aggregator's interests. We observe using 4 empirical traces of mobile social applications, as well as large human mobility dataset, that most of the advertising is associated with a small set of keywords and places, and that this revenue can be obtained even with new privacy guarantees.

