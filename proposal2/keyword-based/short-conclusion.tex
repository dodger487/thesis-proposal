\section{Conclusion}
\label{sec:conclusion}

The collection and monetization of location information has become a large concern. % for privacy advocates and regulatory bodies. 
The main contribution of this paper is the design and analysis of a solution for location privacy using economics.
Our solution is simple -- opt-in users decide which locations to reveal and only these locations are sold on an information market. 
Buyers pay to gain access to users at specified locations.
Locations are specified in keywords, a notion intuitive to both end users and advertisers.
Our solution relies on a privacy protection component that ensures that the location information the user chooses not to release will not be leaked, and also minimizes the linkage of the user's identity with the released information. 
Future research  directions on keyword-based disclosure may include reducing the role of the trusted third party, larger implementations, and a stronger economic analysis of the solution.
A few locations, at a cell level, have been shown to provide poor anonymity~\cite{de2013unique}.
An interesting open question is if keywords provide better k-anonymity.

% We find that in terms of the value of location information, very few locations (5\%) account for majority of the value.
% We find that potentially sensitive locations (as defined by sociological research) appear to be well distributed across locations sorted by popularity and profitability.
% Likewise, we find that the potential revenue of these sensitive locations is small compared to the total value generated from all locations. 
% This suggests a sweet-spot between location privacy and monetizing location information.
% We construct and deploy a small scale version of the system with real users, showing that our solution is indeed feasible.
% We observe their behaviors and lay the groundwork for future study.
