% Introduction

% The primary goal of big data is to guide decision making.
The online economy is based primarily on advertising.
The income of a firm roughly translates to (number of impressions) x (dollars per impression).
I am trying to keep this abstract and not saying that firms are always getting paid for impressions, as other models like paying per click or per sale or other action are quite common.
Really the argument here is that firms make money based on how many people come to their site and how well they can target advertisements to those individuals.
This gives firms an incentive to gather as much information about their users as possible so that they can better target ads to them.

% sec:background
% This can of course present a privacy issue.
This framework presents a challenge to privacy.
User information is collected and gathered in one centralized place.
There are multiple risks involved here: the firms themselves may use the information in ways the users disagree with, the firms may sell or be coerced to give their information to other firms or governments, or the firms may fall victim to cybersecurity attacks, leaking information to other sources.
TODO: cite some stuff.

As ways to counter this, schemes have been proposed to encrypt user behavior and information, denying all access to a firm.
However, this would deny firms the ability to make money, meaning no services would be provided for users and possibly a lower global utilty be reached.
Thus, schemes that ignore this economy however are unlikely to be adopted.
Companies need to make money to function.
Currently, users seem happy to provide their data in exchange for free services.
A concern is that users do not have a good idea of their data and do not know how it is being used and to whom it is accessible.

Therefore it is important to gain an understanding of how users value their information, what they believe firms are doing with their data, and what users are comfortable with in terms of data use.


\subsection{Related Work}


\subsection{Completed Work}

How does one determine how a study participant values something as abstract as 
In our work, ``Your Browsing Behavior for a Big Mac", TODO(cite) we utilized 
% As demonstrated 

\begin{itemize}
  \item Your Browsing Behavior for a Big Mac (maybe?)
  \item For Sale: Your Data. By: You (some portion?)
  \item Challenges of Keyword-Based Location Disclosure
\end{itemize}

``Your Browsing Behavior for a Big Mac"
User privacy is extremely important.
However, there does not exist a strong understanding of how users value their privacy.




