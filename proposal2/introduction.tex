TODO

\subsection{Outline}
I will begin with a background section which introduces the core concepts found in this proposal: location data, privacy, and bias.
I proceed with three chapters detailing completed work.
Each chapter contains a section summarizing relevant prior work.

\chap{sec:econ} focuses on location data, privacy, and economics.
We begin with work that seeks to understand user attitudes to their privacy and the economic value of their information.
Specifically, it examines an alternative to the current practice of firms offering free services in exchange for full control over user data.
The alternative model is one in which users control their data and make decisions about when to sell access to their info, and to whom.

% Linking Users Across Domains with Location Data
% FindYou
\chap{sec:anon} examines the possibility of anonymizing location data.
Prior work has shown that users are highly unique in their location patterns, leaving them vulnerable to deanonymization (see~\ref{ssec:privacy}).
Here we take this a step further, showing not only that this vulnerability exists, but that users indeed can be linked to other datasets.
Additionally, we provide a tool to users that aggregates and displays their location data along with the potential inferences made from it.


% I Don’t Have a Photograph But You Can Have My Footprints
% Under submission work on demographic labeling
% Current work on bias!
\chap{sec:bias} shows the potential for location data to be part of systems that 
We gather a dataset of locations attached to demographic information from a popular image-sharing mobile application.
This data allows us to study the differences in human mobility across different groups, and moreover, to show that demographics can be inferred using only location data.
This raises questions about the sensitivity of location data, and about the potential for bias in systems that make decisions based on location data.
We examine other methodologies for inferring demographics from social network data and discuss debiasing of algorithms.

\chap{sec:proposal-i}

\chap{sec:proposal-ii}

I conclude with a plan for completing this work in \chap{sec:plan}.

