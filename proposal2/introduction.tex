% Ermergerd big data
The era of Big Data has the potential to transform all aspects of society, from business operations to individual leisure time, from medicine to elections.
As the cost of computation and data storage has fallen, and as more human behavior has moved to easily-recorded digital mediums, both public and private institutions have captured and stored more and more information.
``Data is the new oil", analysts have proclaimed, as businesses seek to become more efficient or create innovative digital products. % TODO: cite
``Data will power the cities of tomorrow" governments have proclaimed, releasing their troves of data to the world.

% Shift: what are the problems
The excitement of the potential gains has been tempered by many potential concerns.
% Anonymity: NSA
With the desire for public agencies that could more accurately prioritize safety concerns has come massive surveillance, both by companies and governments, raising concerns of \textbf{anonymity}. % TODO: cite
% Economics: price discrimination
With the belief that Big Data would lower costs and benefit consumers has come evidence of large-scale price discrimination~\cite{wsj, Anonymous:2012wi}, with its impacts possibly most felt by already economically vulnerable populations, raising concerns of \textbf{economics}.
% Algorithmic bias: bail-setting
And with the hopes of making government more fair and efficient have come allegations that seemingly impartial algorithms actually encode unfair racial biases~\cite{propublica:bias}, raising concerns of \textbf{algorithmic bias}.
We must carefully consider how to obtain the benefits of the Big Data Age without paying too high a price.

% Location data
% Although Big Data encompasses the massive colletion of information from all areas of human activity, of particular interest and concern is data generated by the everyday activity of human beings, as this 
One particularly interesting and sensitive subset of Big Data is information relating to the real world movements of individuals: ``location data".
As this data is tied to actual physical movements, it is of enormous interests to businesses and governments.
When people move, they are expending energy and money to do so, as opposed to the many relatively low-cost digital behaviors captured online.
Thus, location data provides a valuable input for many problems, as evidenced by academic proof-of-concepts and by the many businesses and services surrounding it's capture, such as Google Location Services, Foursquare\footnote{\url{foursquare.com}}, Placed\footnote{\url{placed.com/}}, and many more.

% Problem: how to use location data safely?
% TODO: explain a bit more about what this case was about?
Along with its valuable signal again comes concerns of safety and privacy.
In the United States v. Jones decision, in which the Supreme Court wrestled with the legality of law-enforcement placing a GPS tracker on a car without a warrant, Justice Sotomayor wrote 
``disclosed in [GPS] data ... [are] trips the indisputably private nature of which takes little imagination to conjure: trips to the psychiatrist, the plastic surgeon, the abortion clinic, the AIDS treatment center, the strip club, the criminal defense attorney, the by-the-hour motel, the union meeting, the mosque, synagogue or church, the gay bar and on and on"~\cite{jones2012us}.
Clearly, if location data is to be used by analysts, it must be done so in a way that is extremely careful and respectful of user preferences.
% TODO: add another sentence

In this thesis proposal, I examine work that attempts to reconcile the benefits of data-mining location data with three key areas of concern: anonymity, economics, and algorithmic bias.
I will begin with a background section which introduces the core concepts found in this proposal, continue with several chapters relating to the afore-mentioned areas of concern, and conclude with a final discussion of next steps in making big location data safe and fair to use.

\section{Anonymity}
% Linking Users Across Domains with Location Data
% FindYou
\chap{sec:anon} examines the possibility of anonymizing location data. %TODO: edit
Prior work has shown that users are highly unique in their location patterns, leaving them vulnerable to de-anonymization (see~\ref{sec:privacy}).
Here we take this a step further, showing not only that this vulnerability exists, but that users indeed can be linked to other datasets.
Additionally, we provide a tool to users that aggregates and displays their location data along with the potential inferences made from it.

\section{Economics}
I focus on location data, privacy, and economics in \chap{sec:econ}.
We begin with work that seeks to understand user attitudes to their privacy and the economic value of their information.
Specifically, it examines an alternative to the current practice of firms offering free services in exchange for full control over user data.
The alternative model is one in which users control their data and make decisions about when to sell access to their info, and to whom.


\section{Algorithmic Bias}
% I Don’t Have a Photograph But You Can Have My Footprints
% Under submission work on demographic labeling
% Current work on bias!
Finally, I analyze the interaction between location data and algorithmic bias in \chap{chap:bias}.
We gather a dataset of locations attached to demographic information from a popular image-sharing mobile application.
This data allows us to study the differences in human mobility across different groups, and moreover, to show that demographics can be inferred using only location data.
This raises questions about the sensitivity of location data, and about the potential for bias in systems that make decisions based on location data.
We examine other methodologies for inferring demographics from social network data and discuss debiasing of algorithms.
Furthermore, I demonstrate a personal auditing tool for users to understand their location data and the inferences potentially made about them.

\section{De-biasing Location Based Advertising}
\chap{chap:proposal} links my previous works together in my proposal for future work.
I hope to discover the conditions under which adveritisers may safely use location data to decide when to show ads without unfairly benefiting one demographic group at the cost of another.
% This work will require an understanding of the 
Advertisers (and other users of algorithms) often wish to target users based on certain traits (e.g. interest in a product).
These users may be clustered geographically, for example, only users in New York will be interested in purchasing a subway pass for that city.
While wishing to target, advertisers may also wish to \emph{not} exclude certain demographics due to issues of fairness (and potential bad publicity).
Our New York City subway pass advertiser might not wish to offer more deals to the wealthy at the expense of the poor, for example.
By using a large dataset of location data tagged with user generated text and demographic information, I will be able to determine the cost of altering a location-based advertising algorithm to insure that ads are shown to a representative population.
A key metric will be demographic mobility, looking at how skewed from the overall distribution the make up of visits by certain demoagrphics are in a location.
In locations where the difference in demographic mobility is small, advertisers can tweak their algorithms to show ads equally across demographics with minimal cost, fair to both individuals and overall groups.
In other cases where the difference is large, the costs will likely be higher.
I offer more detailed description of this planned work in \chap{chap:proposal}.

% I discuss my plans to analyze real world algorithmic bias in a location-based advertising setting along with my current progress.

% \chap{sec:proposal-ii}

% I conclude with a plan for completing this work in \chap{sec:plan}.

