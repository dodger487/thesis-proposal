The ``Big Data" era has a lot of potential.
Businesses hope that Big Data will make them more efficitient and profitable or enable entirely new products.
Governments hope to provide better for the needs of their citizens.
At the same time, these new collections of data can present societal risks, as we've now enabled mass surveillance, a loss of privacy, and algorithmic bias.

The rise of cheap and ubiquitous electronics, including but not limited to smartphones, has enabled the capture and use of human mobility data like never before.
As a subset of Big Data, location data can be a boon to both profit centers and scientific understanding, but comes with many risks attached.
The places we visit can reveal much about ourselves, whether proclivities towards particular type of food, or more private characteristics of race, religion, sexuality, or political affiliation.

In this thesis proposal, I describe recent work that attempts to balance the scientific and engineering promises of location data with the potential risks.
I will describe work I have completed relating location data to privacy, anonymity, economics, and algorithmic bias.
I propose future research to be completed in the form of a thesis, advancing knowledge of location-based demographics and algorithmic bias.
