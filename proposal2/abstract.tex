% Big data!! Accompanied by risks and benefits
The ``Big Data" era has begun, bringing with it a host of possibilities and concerns.
The ability to store and process records of minute behavioral details about billions of people will hopefully lead to more efficient and effective businesses, governments, and organizations.
At the same time, these new collections of data present societal risks, enabling mass surveillance, a potential loss of privacy, and the capability to computationally discriminate at massive scale.

% Location data, a subset of big data.
A rich subset of this data is human mobility data: records at an individual scale detailing where and when someone moves.
This data is now captured like never before due to the rise of smartphones and other cheap and ubiquitous electronic devices.
This location data can be a boon to both profit centers and scientific understanding, but comes with many risks attached.
The places we visit can reveal much about ourselves, whether proclivities towards particular type of food and hobbies, or more private characteristics of race, religion, sexuality, and political affiliation.
As organizations begin to harness this data, it is clearly important to make sure that such information is used in a way that reduces potential harms to individuals or vulnerable groups.

% Description of completed work and proposal
In this document, I describe recent and in-progress work that attempts to balance the scientific and engineering promises of location data with the potential risks, looking at three classes of problems.
I begin with my work on anonymity, examining when location data retains an identifiable ``fingerprint" of a user that can be linked to data sets generated by completely separate behaviors, making anonymization difficult or impossible.
I continue with work focusing on privacy and economics, aiming to reconcile the economic incentives for firms to collect location data with usable user choice and a better understanding of users' beliefs and desires in relation to data capture.
Next, I examine the relationship between location data and algorithmic bias, showing that location data can be used to infer sensitive traits and developing a tool to inform users about what their data may be revealing.
% I conclude with a proposal to research the algorithmic bias that may be inherent in location-based advertising systems, combining millions of locations collected from social media, computer vision, and the state-of-the-art results in algorithmic de-biasing.

I conclude with my proposal to conduct an analysis that will show the conditions in which a firm may engage in location-based advertising without unfairly distributing their offers across different demographics.
This work will combine a dataset of millions of locations collected from social media, computer vision techniques, and state-of-the-art results in algorithmic de-biasing to show, for the first time, the trade off between revenue and fairness in a location-based advertising setting.

% business - profits
% ngo - effectiveness
% govt - efficient
