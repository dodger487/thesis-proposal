Ubiquitous, mobile computing in the form of smartphones has created data that lets us study human behavior like never before.
In particular, data about human mobility has allowed us to understand the hows and whys of human movement.
At the same time, these new collections of data can present societal risks, as we've now enabled mass surveillance, a loss of privacy, and algorithmic bias.

In this thesis proposal, I describe recent work that attempts to balance the scientific and engineering promises of location data with the potential risks.
I will describe work I have completed relating location data to privacy, anonymity, economics, and algorithmic bias.
I propose future research to be completed in the form of a thesis, advancing knowledge of location-based demographics and algorithmic bias.
