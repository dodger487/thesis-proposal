Human mobility is intimately intertwined with highly personal behaviors and characteristics. As Justice Sotomayor of the United States Supreme Court stated, ``disclosed in [GPS] data ... [are] trips the indisputably private nature of which takes little imagination to conjure: trips to the psychiatrist, the plastic surgeon, the abortion clinic, the AIDS treatment center, the strip club, the criminal defense attorney, the by-the-hour motel, the union meeting, the mosque, synagogue or church, the gay bar and on and on~\cite{jones2012us}.'' For that reason, previous studies of mobility centered on the risk of either re-identification in sensitive anonymized location datasets or on protecting visits to private locations~\cite{de2013unique,Guha:2012ws}.

However, the re-identification risk based on individual locations is not the only threat. Many users are producing a series of footprints, which might be innocuous individually, however, taken together can create a sparse yet informative view allowing inferences from their whereabouts. The benefits of revealing locations are obvious: location data can be used for personalizing recommendations~\cite{ICWSM112886} and displaying more relevant advertising~\cite{Lindamood:2009:IPI:1526709.1526899} in order to finance free online services. However, the downsides are more difficult to assess. While an individual data point may create no privacy risk, an aggregated dataset might enable inferences beyond a user's expectation.

In this paper we explore the discriminative power of location data. Solely based on mobility patterns, which we extracted from photosharing network profiles, we infer users' ethnicities and gender both on a demographic and an individual level. As we discuss in \S\ref{sec:relwork}, this exploration stands in contrast to limitations of previous studies as our paper brings together the following contributions:
\begin{itemize}
  \item We show how photosharing network data can be leveraged to extract mobility patterns using a new method for creating location datasets from publicly available resources. Our method combines the use of online social networks and crowdsourcing platforms. It has the advantage that it generally enables \emph{anyone} to study human mobility and does not mandate access to Call Detail Records (CDRs) or other proprietary datasets. (\S\ref{sec:method}).
  \item To assess the quality of the created datasets we show that mobility patterns extracted from photosharing networks are comparable in terms of their essential characteristics to those previously observed and reported for CDRs. For the first time, we extend the analysis of mobility patterns to \emph{ethnic groups}. We show how comparisons lead to statistically significant differences that are meaningful for assessing residential and peripatetic segregation. (\S\ref{sec:patterns}). 
  \item Finally, we demonstrate the discriminative power of location data on an \emph{individual} level. Our analysis confirms for the first time that location data alone suffices to predict an individual's ethnicity, even with relatively simple frequency-based algorithms. Moreover, this inference is robust: a small amount of location records at a coarse grain allows for an inference competitive with more sophisticated methods despite of data sparsity and noise. (\S\ref{sec:inference}).
\end{itemize}

% Our study opens multiple avenues of research made possible by informative and publicly available location data, which we summarize along with our results in (\S\ref{sec:conclusion}).
