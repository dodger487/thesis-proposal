% Location data is increasingly available and its use holds many risks and opportunities.
This study highlights the risks and opportunities of discriminative big data analysis by demonstrating that it is possible to infer Internet users' ethnicities and genders based on location data \emph{alone}. It also shows that mobility patterns can be studied using publicly available data. Internet users may often be unaware that releasing such data could also disclose possibly sensitive personal information. Simply reducing granularity proved to be insufficient to prevent such privacy leakage as mobility remains discriminative. However, the trove of geotagged pictures available through individual online profiles also yields important insights for beneficial uses, for example, by city planners and social scientists. 

As our dataset is similar, both demographically and mobility-wise, to other datasets as shown in \S\ref{sec:patterns}, we believe that our results are generalizable and applicable to other unlabeled datasets. Although it could be claimed that our data is biased by the fact that the users in our study have willingly disclosed their gender and ethnicity by publicly using Instagram, we want to stress that it would be difficult and possibly unethical to create a labeled dataset of users who \emph{do not} want to disclose their gender and ethnicity.

This work motivates multiple avenues of further research: First, it enables the extension of demographic mobility analysis to many researchers using shareable public datasets and reproducible results. Beyond ethnicity and gender, attributes such as age, occupation, and other lifestyle features may be extracted from users' pictures, and naturally there are many other mobility properties to account for beyond, for example, daily ranges. Second, better understanding the discriminative power of location data might inform the design of tools for raising user awareness about the information they reveal. This insight motivates revisiting mobility modeling and the inferences it renders possible to empower users to make at will their locations as clear as a photograph or as opaque as footprints in the mud. 

% , showing also the impact of granularity, location anonymization, and data quantity on this inference.
% Additionally, we show statistically significant differences in mobility at a population level between gender and ethnic groups.
% We first demonstrated that geotag data from Instagram reproduces and accurately represents both mobility behaviors from a prior study and demographic patterns from the U.S. Census.
% We hope that our methodology will be useful for others hoping to obtain location data labeled with demographic information.
% All data used in the study will be made available in an anonymized fashion upon publication.

% New m

% Location data is increasingly available and its use holds many risks and opportunities.
% As smartphones become more affordable and ubiquitous, more people will be regularly 
% % As we've shown, 
% They may not realize that releasing location data can also reveal information about their gender and ethnicity
