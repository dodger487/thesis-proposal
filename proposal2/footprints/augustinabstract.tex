Location data are routinely available to a plethora of mobile apps and third party web services. The resulting datasets are increasingly available to advertisers for targeting and also requested by governmental agencies for law enforcement purposes. While the re-identification risk of such data has been widely reported, the \emph{discriminative} power of mobility has received much less attention. In this study we fill this void with an open and reproducible method. We explore how the growing number of geotagged footprints left behind by social network users in photosharing services can give rise to inferring demographic information from mobility patterns. Chiefly among those, we provide the first detailed analysis of \emph{ethnic} mobility patterns in two metropolitan areas. This analysis allows us to examine questions pertaining to spatial segregation and the extent to which ethnicity can be inferred using \emph{only} location data. Our results reveal that even a few location records at a coarse grain can be sufficient for simple algorithms to draw an accurate inference. Our method generalizes to other features, such as gender, offering for the first time a general approach to evaluate discriminative risks associated with location-enabled personalization.
