Our study complements works on human mobility patterns and attribute inference in multiple ways. 

% \subsection{Human Mobility Patterns}
% \label{subsec:patterns}
% 
First, the use of location data relates our study to previous inquiries into human mobility ~\cite{Cho:2011io,Gonzalez:2008wy,ICWSM112831}. In particular, we aggregate location data into mobility patterns and compare our patterns to those published in earlier studies~\cite{Becker:2013ci,Isaacman:2011cn,Isaacman:2010en} for validation, but furthermore we analyze those patterns both at an individual level and aggregated in multiple demographic groups, including, for the first time, from the perspective of ethnicity. This analysis complements previous studies which have shown that mobility is correlated to social status~\cite{ICWSM112783} and community well-being~\cite{LathiaQC12} measured at city and neighborhood levels. 
While some studies already demonstrated that mobility traces can uniquely identify individuals \cite{de2013unique,song2010limits}, the inference of individuals' demographic attributes from location data, that is, the \emph{discriminative} power of location data, remained unexplored. 
%However, different from earlier studies we are not only interested in revealing mobility patterns, but we also want to determine their inferential power. 
% We want to infer individuals' characteristics from location data.
We make inferences beyond trip purpose identification~\cite{doi:10.1061/41123(383)73}, activity type prediction~\cite{liao07extracting,liu13annotating}, and identification of location types~\cite{Isaacman:2011un}. 

% \subsection{Attribute Inference}
% \label{subsec:inference}
% 
Previous studies aimed to infer the ethnicities, gender, and other attributes of online users. Often they leveraged linguistic features, such as Facebook or Twitter user names, stated first and last names~\cite{ICWSM101534,mislove-2011-twitter}, or Tweet content~\cite{ICWSM112886,Rao:2010:CLU:1871985.1871993}. Those studies demonstrated an underrepresentation of females and minorities online~\cite{mislove-2011-twitter}; a finding which we extend and confirm using photosharing services. Mobility data from mobile phones were used to predict personality traits~\cite{deMontjoye:2013:PPU:2456485.2456492}, age~\cite{Brea:el}, and gender~\cite{Sarraute:2014ka}, but, in addition to relying on proprietary data, all of these studies solely analyzed call patterns or social network properties as opposed to locations. In contrast, we attempt to infer attributes using \emph{only} location data, making our work more broadly applicable to any technology that can collect mobility information, such as GPS, Wi-Fi, or mobile apps. We additionally examine whether predictions become more accurate with more data, similar to~\cite{AltshulerAFEP12}, and how the granularity of data impacts prediction accuracy.

% It is well established that linguistic features, such as Facebook user names ~\cite{ICWSM101534} or Tweets ~\cite{Rao:2010:CLU:1871985.1871993,ICWSM112886}, can be leveraged to infer social network users' ethnicities, genders, and other attributes. Various studies demonstrated that CDRs are useful for such inferences as well. In particular, using CDRs ~\cite{liu13annotating} showed that it is possible to infer activities, e.g., going out for a social visit. ~\cite{deMontjoye:2013:PPU:2456485.2456492} further demonstrated the viability of CDRs for predicting personality traits. Similar to the aforementioned studies we aim to infer social network user attributes, more specifically, ethnicity and gender. However, in contrast to those studies we want to evaluate the inferential power of location data: Are locations \emph{by itself} sufficiently discriminative in the sense that they allow the accurate prediction of ethnicity and gender? In order to answer this question we examine, among others, whether predictions become more accurate over time with more data (e.g., ~\cite{conf/socialcom/AltshulerAFEP12}) and how the granularity of data impacts prediction accuracy.

% \subsection{Online Social Networks}
% \label{subsec:networks}
% 
More generally, our analysis fits into the category of works on extracting information from social networks, such as \cite{Cranshaw:2010:BGP:1864349.1864380}. Probably, the closest work is~\cite{Zhong:2015:YYG:2684822.2685287}, which also aims to infer meaning from locations, however, is not concerned with ethnicity. We obtain our data from profiles of the photosharing service Instagram, and our analysis is enhanced with auxiliary information from the geo-social search service Foursquare and the United States Census 2010~\cite{census:2010} (Census). To our knowledge this is the first study demonstrating that it is possible to extract from social networks mobility patterns that are enriched with ethnic or gender information at an individual level. It should be noted in particular that all aforementioned studies of mobile data rely on proprietary data, primarily CDRs, that are only available with the consent of the data owner (e.g.,~\cite{de2013unique,LathiaQC12}). In contrast, our methodology is principally reproducible by anyone at a small cost, and our data will be made available shortly after publication. Our study provides a contribution to overcome the lack of publicly available mobility datasets and serves as a validator for their patterns.
